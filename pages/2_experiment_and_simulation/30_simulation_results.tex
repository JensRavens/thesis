\subsection{Comparing the simulated system to the original data}

To create reusable algorithms for the simulated data, the concept of a slice is introduced. A slice is a two dimensional datastructure holding the value of the enhancement using $x$ and $y$ coordinates. These coordinates reflect the $x$ and $y$ coordinates of the overal simulation but don't have to be a plane in the $z$ plane. Instead slices can be any geometric structures representable as $z=f(x,y)$. The local enhancement inside slices is only accounting for an electric field $\mathbf{E}$ that is parallel to the slice surface. For slices along the $z$-axis this means discarding the $z$-component of the field, for more complex surfaces the field has to be projected to that surface for every value.

In figure \ref{fig:slices} there are the plotted versions of slices using equidistant planes along the $z$-axis to give an overview over the enhancement at different values of $z$. Noticeable are the fragments of spatial discretisation along the rounded gold structures.

Figure \ref{fig:slices} shows the resonance of plasmons with light polarized in $x$-direction. The cavity between the gold nanostructures the biggest amount to the total enhancement as already shown by \cite{heeg}. Above the structure ($z>\SI{45}{nm}$) the enhancement drops off significantly.
