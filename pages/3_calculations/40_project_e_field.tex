Now that the position of graphene is known for every point in the $xy$ plane the electric field can be projected onto this plane. This projection yields another slice, now in the plane of graphene:

\begin{minted}{matlab}
  x_length = length(obj.x);
  y_length = length(obj.y);
  grid_size= 0.5;
  data = ones(x_length, y_length);
  for x_loop=1:x_length
     for y_loop=1:y_length
         height = map.height(x_loop, y_loop);
         [~, index] = min(abs(obj.z - height));
         sliceX = obj.Ex(x_loop,y_loop,index);
         sliceY = obj.Ey(x_loop,y_loop,index);
         sliceZ = obj.Ey(x_loop,y_loop,index);
         height_next_x = map.height(x_loop+1, y_loop);
         height_next_y = map.height(x_loop+1, y_loop);
         x_projection = cos(atan((height_next_x - height) / grid_size));
         y_projection = cos(atan((height_next_y - height) / grid_size));
         z_projection = 1 - cos(atan((height_next_y - height) / grid_size));
         data(x_loop, y_loop) = sliceX .* conj(sliceX) * x_projection +
            sliceY .* conj(sliceY) * y_projection +
            sliceZ .* conj(sliceZ) * z_projection;
     end
  end
  graphene_slice = slice(data, obj.x, obj.y);
\end{minted}

The returned slice was just a regular slice which was plotted as seen in figure \ref{fig:projected}a. The projected near field enhancement for the 3D-geometry of graphene looks suprisingly similar to a projection on a planar slice at a hieght of 40 nm as seen in figure \ref{fig:projected}a and figure \ref{fig:projected}b.

\begin{figure}[!h]
  \centering
  \begin{subfigure}[t]{0.48\textwidth}
    \caption{}
    \includegraphics[width=\textwidth]{./images/graphene-layer.png}
  \end{subfigure}
  ~
  \begin{subfigure}[t]{0.48\textwidth}
    \caption{}
    \includegraphics[width=\textwidth]{./images/40nm-2nm.png}
  \end{subfigure}
  \caption{\textbf{(a)} Near field enhancement $|E/E_0|^2$ in the layer of graphene after applying the height map. \textbf{(b)} Near field enhancement of a planar slice at a height of $z=\SI{40}{nm}$ for comparision.}
  \label{fig:projected}
\end{figure}
