\subsection{Raman Spectroscopy}

The inelastic scattering of light is referred to as Raman scattering. When photons exchange energy with the material during a scattering process they can either lose energy (Stokes process) or gain additional energy (anti-Stokes process). This energy in the material is vibrational energy, also known as phonons. The process of transferring energy can also be described as
the inelastic collision of a photon with a phonon.

By gaining or losing energy the photon changes it's frequency. This shift in the photon's
frequency is referred to as the Raman shift or Raman frequency. The Stokes shifted frequency is

\begin{equation}
  \omega_S=\omega_i-\omega_0
\end{equation}


with $\omega_i$ being the frequency of the incoming light and $\omega_0$ the phonon frequency.
The anti-Stokes shifted frequency is shifted by the same amount $\omega_0$ as

\begin{equation}
  \omega_{AS}=\omega_i+\omega_0.
\end{equation}
