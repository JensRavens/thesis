\subsubsection{Snapshots and Transforms to a Frequency Domain}

While it is often useful to know the exact state of the system at any time step $q$, this actually leads to a few problems in practical implementations:

\begin{description}
  \item[Memory Consumption] Assuming that each value is a complex value with double precision leads to a space consumption of $2\cdot64$bit = $128$bit or $16$byte for every single point of $\mathbf{E}$ and $\mathbf{H}$. A cubic system with a side length of $200nm$ and a grid size of $0.5nm$ already has 64 million points which corresponds to about $976$mb of memory consumption just for the $\mathbf{E}$ field for a single point in time.
  \item[Mixed $\mathbf{E}$ and $\mathbf{H}$ Fields] If the discretisation $\Delta_t$ is not corresponding to the frequency of the propagation of the electromagnetic wave there will be a mixture of $\mathbf{E}$ and $\mathbf{H}$ at any given timestep $q$ which is undesired in most cases because only the $\mathbf{E}$ field might be of interest.
\end{description}

To solve these problems most implementations of FDTD transform the electric field $\mathbf{E}(\mathbf{x},t)$ with a fourier transform to $\mathbf{E}(\mathbf{x}, \nu)$ with $\nu$ being the frequency of the electric field. In case only a specific frequency $\nu$ is of interest all other values can already be discarded while the simulation is running which is greatly reducing the memory needed to perform a calculation.

This also solves the problem of mixed fields: Over many time steps $q$ the calculated $\mathbf{E}$ values will average out the error that would be present by just using a single iteration.
