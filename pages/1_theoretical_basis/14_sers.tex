\subsection{Surface Enhanced Raman Spectroscopy with Graphene}

Plasmonics explores the interaction of metallic surfaces and light waves that causes density waves of electrons on the surface. The electron density wave propagating on the metal is often referred to as a surface plasmon. \note{mention localized plasmons}

\begin{figure}[!h]
  \centering
  \begin{subfigure}{0.45\textwidth}
    \includegraphics[width=\textwidth]{./images/sers-schema.png}
  \end{subfigure}
  ~
  \begin{subfigure}{0.45\textwidth}
    \includegraphics[width=\textwidth]{./images/local-enhancement-heeg.png}
  \end{subfigure}
  \caption{\textbf{(a)} Gold disks are placed with a chrome interlayer on SiO$_2$. A layer of graphene is pulled into the gap. Adapted from \cite{heeg}. \textbf{(b)} Local enhancement at $\lambda = 638nm$ and $z=40nm$. The blue dashed line indicates the area that includes 90\% of the near field intensity according to \cite{heeg}. Copied from \cite{heeg}.}
  \label{fig:heeg-experiment}
\end{figure}

\note{komponente des E-felds parallel zum raman dipolmoment erzeugt verstärkung, für graphen: nur Komponente des Nahfelds in der Graphen Ebene zählt}
