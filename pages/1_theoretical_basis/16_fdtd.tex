\subsection{Finite-difference time-domain method}

The finite-difference time-domain method (also called Yee's method after the mathematician Kane Yee) describes an algorithm to calculate time dependent electromagnetic fields based on Maxwell's equations. It has first been published in a paper by Kane Yee in 1966\cite{yee} (\question{here was an unclear comment in the pdf}).

To describe a system, space and time are broken down into a grid (which can but does not have to have equidistant points) which contains information about the electric and magnetic fields and the permittivity at each point. After the grid is initialized, the neighboring electric field values can be calculated from the magnetic field and - in a second step - the future electric field can be calculated from the magnetic field. By taking the permittivity into account this method is capable of describing inhomogenous complex systems that consist of different materials and field sources.

While there are other often more efficient methods to numerically simulate and calculate surface plasmons, the finite-difference time-domain method is way more flexible. Because it is basically just a solver for Maxwell's equations without any assumptions on the geometry it can calculate various experiments from wave guides to all kinds of light interaction with wavelength-scale 3D metal geometries\cite{numel}.
