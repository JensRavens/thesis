\newpage
\section{Conclusion and Outlook}

This work described the simulation of an experiment conducted in \cite{heeg}. Graphene was used as a Raman active material in surface enhanced Raman spectroscopy on a gold nanostructure. Because the laser's width is way bigger than the cavity the calculations for the cavity cannot directly be compared to the experimental results.

It was shown that by using a more accurate method of measuring the enhancement along the graphene layer the results only slightly differ from the paper's assumption that the field in the $z=\SI{40}{nm}$ plane gives the biggest part in the total enhancement.

In the last part the spatial coherence was taken into account which leads to a final enhancement value of $23.0$ which is pretty close to the measured value of $12.8$ in \cite{heeg}.

The corner radius of the gold structure only has a minor influence on the total enhancement, although it leads to significant changes in the $x,z$-plots of the near-field. Further simulations should include even bigger corner radiuses to find an experimental setup that results in the same near-field at the edges of the nanostructure.
