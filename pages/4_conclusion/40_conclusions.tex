\newpage
\section{Discussion and Concolusion}

This work described the simulation of an experiment conducted in ref.~\cite{heeg}. Graphene was used as a Raman active material for surface-enhanced Raman spectroscopy with a gold nanodimer as a plasmonic nanostructure. Because the laser spot size is much bigger than the area of the dimer cavity, the calculated local enhancement cannot be directly compared to the experimental enhancement.

It was the scope of this work to draw a comparison between the local simulated enhancement and the total measured enhancement. The shape of the graphene layer, that was suspended on top of the gold nanodimer, was simulated with a 3D modelling software. The local electromagnetic enhancement was calculated with the finite-difference time-domain method, using the commercial software Lumerical FDTD solutions. By averaging the local electromagnetic enhancement at the graphene layer over the size of the laser spot the total enhancement was calculated.

The calculated local enhancement within the simulated area was $60.2$  both for the \SI{2}{nm} corner size and the \SI{5}{nm} corner size (both in the plane of graphene) and $59.2$ for the \SI{5}{nm} corner size for a planar slize at a height of $z=\SI{40}{nm}$. Taking the laser geometry into account the total enhancement was calculated as $38.1$ for the \SI{2}{nm} corner radius, $37.8$ for the \SI{5}{nm} corner radius and $37.2$ for the \SI{5}{nm} planar slice. Using the spatial coherence theory of ref.~\cite{coherence} the enhancement of the \SI{5}{nm} planar slice went down to $23.0$. The corner radius of the gold nanostructure only had a minor influence on the total enhancement, although it led to significant changes in the $xz$ plots of the near-field.

The simulated local electric field enhancement nicely reproduced the simulation of ref.~\cite{heeg} and the resulting total enhancement of the projected graphene layer in comparision to a slice at the height of \SI{40}{nm} confirmed the approximation in ref.~\cite{heeg}. Taking spatial coherence theory into account, the resulting total enhancement of $23.0$ endorsed the measured value of $12.8$ in ref.~\cite{heeg}.

Further simulations should include even bigger corner radiuses to find an experimental setup that results in the same near-field at the edges of the nanostructure.
